
\documentclass[11pt,a4paper,sans]{moderncv}        % possible options include font size ('10pt', '11pt' and '12pt'), paper size ('a4paper', 'letterpaper', 'a5paper', 'legalpaper', 'executivepaper' and 'landscape') and font family ('sans' and 'roman')


% moderncv themes
\moderncvstyle{casual}                            % style options are 'casual' (default), 'classic', 'oldstyle' and 'banking'
\moderncvcolor{grey}                              % color options 'blue' (default), 'orange', 'green', 'red', 'purple', 'grey' and 'black'
%\renewcommand{\familydefault}{\sfdefault}         % to set the default font; use '\sfdefault' for the default sans serif font, '\rmdefault' for the default roman one, or any tex font name
%\nopagenumbers{}                                  % uncomment to suppress automatic page numbering for CVs longer than one page

% character encoding
\usepackage[utf8]{inputenc}                       % if you are not using xelatex ou lualatex, replace by the encoding you are using
%\usepackage{CJKutf8}                              % if you need to use CJK to typeset your resume in Chinese, Japanese or Korean

% adjust the page margins
\usepackage[scale=0.75]{geometry}
%\setlength{\hintscolumnwidth}{3cm}
%\setlength{\makecvtitlenamewidth}{10cm}

% personal data
\name{Grant Smith}{}
\address{550 Stoneridge Dr Apt C206}{Lawrence Kansas, 66049}{}
\phone[mobile]{+1~(913)~660~5415}
\email{14.gsmith.14@gmail.com}
\social[linkedin]{grant-smith-895b64a4}
\social[github]{GSmithApps}

\title{Resumé title}


\begin{document}


\newcommand{\hiringmanager}{Hiring Manager}
\newcommand{\position}{Technology Fundamentals Curriculum Writer/Subject Matter Expert}
\newcommand{\company}{Chegg}

\recipient{\hiringmanager{}}{Chegg}
\date{\today}
\opening{Dear \hiringmanager{}}
\closing{Thank you for your time,}
\makelettertitle

I'm writing about the \position{} position at \company{}. This
position excites me because I love creating courses and content --
especially related to technology.  Part of why it matters to me is
because it has had a huge impact on my life and career, and I want
other people to feel the fulfillment and pride that comes from
learning technology. I remember before I was a programmer, when I
saw people have an IDE open, I thought it was so cool, and now
here I am on the other side with an VSCode open all day.

I didn't start out in technology -- I found it on my own, realized
how much I liked it, and the rest was history.  Since then, I have
worked various jobs related to programming and technology, and taught
a data analytics class as well. Furthermore, I've given talks at MeetUps
and created my own educational content on YouTube, GeoGebra, Medium,
and other channels.

I've been a financial quant at Security Benefit for 18 months. While I've learned
a lot, I miss the direct interaction with people and the chance to spark interest
in technology. I want to be in a role that combines my technical skills with
community engagement.  For example, when I taought the data analytics
course, I could feel that I was helping the students grow a passion for
programming and start to feel empowered and that they actually could learn
it.

As a \position{}, I plan to write clear documentation, create straightforward
tutorials, and present at conferences. Listening to customers and relaying their
feedback to our developers is a priority for me. Being a developer, I understand
the technical side and aim to bridge the gap between our team and our users effectively.

Thank you for considering my application. I would be happy to discuss
how I can be a core member of the team at Chegg and help our students
thrive.

\vspace{0.5cm}


\makeletterclosing

\end{document}


%% end of file `template.tex'.
